% Copyright (C) 2021-2022 Diogo Rodrigues
% Distributed under the terms of license
% Creative Commons Attribution-NonCommercial-NoDerivatives 4.0 International

\documentclass{beamer}
% Encodings (to render letters with diacritics and special characters)
\usepackage[utf8]{inputenc}
% Language
\usepackage[english]{babel}
\usepackage{verbatim}

\usepackage{multicol}

\usepackage{setspace}

\usetheme{Madrid}
\usecolortheme{default}

\pdfstringdefDisableCommands{
  \def\\{}
  \def\texttt#1{<#1>}
}

\newcommand{\email}[1]{{\footnotesize\texttt{\href{mailto:#1}{#1}}}}

\usepackage{caption}
\DeclareCaptionFont{black}{\color{black}}
\DeclareCaptionFormat{listing}{{\tiny \textbf{#1}#2#3}}
\captionsetup[lstlisting]{format=listing,labelfont=black,textfont=black}
\usepackage{subfigure}

\usepackage{listings}
\lstset{
    frame=tb, % draw frame at top and bottom of the code
    basewidth  = {0.5em,0.5em},
    numbers=left, % display line numbers on the left
    showstringspaces=false, % don't mark spaces in strings  
    commentstyle=\color{green}, % comment color
    keywordstyle=\color{blue}, % keyword color
    stringstyle=\color{red}, % string color
	aboveskip=-0.2em,
    belowskip=-0.2em,
    basicstyle=\tiny
}
\lstset{literate=
  {á}{{\'a}}1 {é}{{\'e}}1 {í}{{\'i}}1 {ó}{{\'o}}1 {ú}{{\'u}}1
  {Á}{{\'A}}1 {É}{{\'E}}1 {Í}{{\'I}}1 {Ó}{{\'O}}1 {Ú}{{\'U}}1
  {à}{{\`a}}1 {è}{{\`e}}1 {ì}{{\`i}}1 {ò}{{\`o}}1 {ù}{{\`u}}1
  {À}{{\`A}}1 {È}{{\'E}}1 {Ì}{{\`I}}1 {Ò}{{\`O}}1 {Ù}{{\`U}}1
  {ä}{{\"a}}1 {ë}{{\"e}}1 {ï}{{\"i}}1 {ö}{{\"o}}1 {ü}{{\"u}}1
  {Ä}{{\"A}}1 {Ë}{{\"E}}1 {Ï}{{\"I}}1 {Ö}{{\"O}}1 {Ü}{{\"U}}1
  {â}{{\^a}}1 {ê}{{\^e}}1 {î}{{\^i}}1 {ô}{{\^o}}1 {û}{{\^u}}1
  {Â}{{\^A}}1 {Ê}{{\^E}}1 {Î}{{\^I}}1 {Ô}{{\^O}}1 {Û}{{\^U}}1
  {Ã}{{\~A}}1 {ã}{{\~a}}1 {Õ}{{\~O}}1 {õ}{{\~o}}1
  {œ}{{\oe}}1 {Œ}{{\OE}}1 {æ}{{\ae}}1 {Æ}{{\AE}}1 {ß}{{\ss}}1
  {ű}{{\H{u}}}1 {Ű}{{\H{U}}}1 {ő}{{\H{o}}}1 {Ő}{{\H{O}}}1
  {ç}{{\c c}}1 {Ç}{{\c C}}1 {ø}{{\o}}1 {å}{{\r a}}1 {Å}{{\r A}}1
  {€}{{\euro}}1 {£}{{\pounds}}1 {«}{{\guillemotleft}}1
  {»}{{\guillemotright}}1 {ñ}{{\~n}}1 {Ñ}{{\~N}}1 {¿}{{?`}}1
}

\usepackage{dirtree}

\usepackage[style=british]{csquotes}

\usepackage{tabularx}

\usepackage{svg}
\usepackage{graphicx}
 
%Information to be included in the title page:
\AtBeginDocument{
\title[\#1. Git Flow]{Git Flow}
\subtitle[]{Monitor session \#1}
\author[Diogo Rodrigues]{
  Diogo Miguel Ferreira Rodrigues (\email{diogo.rodrigues@fe.up.pt})
}
\institute[FEUP/LBAW]{Faculty of Engineering of the University of Porto \\ Database and Web Applications Laboratory (LBAW)}
\date[10/11/2021]{10th of November, 2021}
}

\begin{document}
\frame{\titlepage}

\begin{frame}
  \centering

  \vspace{1em}

  {\Large \bfseries You are not required to know this for LBAW}

  \vspace{1em}

  It will just make your life easier

  with some concepts I wish I knew before LBAW

  \vspace{2em}

  This will not help you learn how to use these specific features,

  rather for you to learn which tools are important
  
  and how they'll help you during your projects
\end{frame}

\begin{frame}
\frametitle{1. Some concepts}
\framesubtitle{1.1. What is git?}

\begin{minipage}{0.49\textwidth}
  Git is a version control program.
  
  \begin{itemize}
    \item Track changes in a set of files (a \textit{repository}, or just \textit{repo})
    \item Collaborate with other people
  \end{itemize}

\end{minipage}%
\begin{minipage}{0.51\textwidth}
  \centering
  \includesvg[width=40mm]{img/git-logo}
\end{minipage}

\vspace{2em}

Main points:

\begin{itemize}
  \item It is just a program in your computer
  \item There is \textbf{no Internet} in the way git works
\end{itemize}

Git contains your whole repository, including past versions.

\end{frame}

\begin{frame}
\frametitle{1. Some concepts}
\framesubtitle{1.2. What is GitHub/GitLab?}

GitHub is a company that remotely hosts git repositories.

{\footnotesize (\textit{Really? Just that?} Basically yes, although they provide some extra services) }

\vspace{0.5em}

GitLab is essentially the same, but with possibility of deploying your own git hosting server.

\vspace{1em}

\begin{minipage}{0.60\textwidth}
  \begin{itemize}
    \item Simplest and most common way to make git distributed and collaborative
    \item Each remote repository is accessible through a link (\textit{origin})
    \item To contribute to a repo:
    \begin{enumerate}
      \item Clone \includegraphics[height=3mm]{img/upload-download}
      \item Edit
      \item Commit
      \item Push \includegraphics[height=3mm]{img/upload-download}
    \end{enumerate}
  \end{itemize}
\end{minipage}%
\begin{minipage}{0.40\textwidth}
  \centering
  \includegraphics[width=15mm]{img/GitHub-logo}\hspace{0.5em}
  \includesvg[width=15mm]{img/GitLab-logo}
\end{minipage}

\end{frame}

\begin{frame}
\frametitle{1. Some concepts}
\framesubtitle{1.3. Tags}

A tag is a human-readable name (an alias) for a particular commit.

\end{frame}

\begin{frame}
\frametitle{1. Some concepts}
\framesubtitle{1.4. Wiki}

Each project in GitHub is associated to a repository and a wiki.

Some things belong with the code: documentation for instance.

What to place in a wiki? Basically everything that does not belong with the code.
\begin{itemize}
  \item Tutorials/guides
  \item Longer documentation
  \item Reports on the process of making the project / how the project works
  \item Website/database architecture
  \item ...
\end{itemize}

The wiki is itself a repo: clone it to use your IDE/editor instead of the online editor

\end{frame}

\begin{frame}
\frametitle{1. Some concepts}
\framesubtitle{1.5. gitignore}

Don't forget to use a \texttt{.gitignore} file.

Files that match the patterns in the \texttt{.gitignore} file will be ignored by git (e.g. binary files)

\end{frame}

\begin{frame}
\frametitle{2. Main subjects}
\framesubtitle{2.1. Branches}

\end{frame}

\begin{frame}
\frametitle{2. Main subjects}
\framesubtitle{2.2. Pull requests}

\end{frame}

\begin{frame}
\frametitle{2. Main subjects}
\framesubtitle{2.3. Git Flow}

Actually I'll be talking about GitHub flow, a simplification of Git Flow.

\end{frame}  

\begin{frame}
\frametitle{2. Main subjects}
\framesubtitle{2.4. Issues}

\end{frame}  

\begin{frame}%[allowframebreaks]
  \scriptsize
  \frametitle{Bibliography}
  \setbeamertemplate{bibliography item}{\insertbiblabel}
  \bibliographystyle{acm}
  \bibliography{report}
\end{frame}

\end{document}
